%=========================================================================
% sec-background
%=========================================================================

\section{Summary of Research on GPGPU Microarchitecture}
\label{sec-background}

In my paper published at ISCA 2013, I focused on further improving the
performance and energy efficiency of regular data-parallel
applications on GPGPUs by exploiting \emph{value structure}, in which
values can be encoded as a compact function of the thread
index~\cite{kim-simt-vstruct-isca2013}.  For example, \emph{affine
  values} are a class of value structure which can be represented in the
following form: $V(i) = b + i \times s$ where $i$ is the thread index,
$b$ is the base, and $s$ is the stride.  Three techniques were developed
to reduce redundant computation by using compactly encoded affine values.
\emph{Affine arithmetic} allows a single arithmetic operation on two
affine values to produce an affine result instead of having every thread
perform a separate operation. \emph{Affine memory operations} eliminate
redundant address generation and coalesce scalar loads without dynamic
address comparisons if the load address is affine. \emph{Affine branches}
allow a single comparison on two affine values to redirect control flow
instead of having every thread perform a separate comparison.  Using
these techniques achieved speedups between 1.2-1.7$\times$ and an
increase in energy efficiency of up to 1.3$\times$ on regular
data-parallel applications on a register-transfer-level model of our
fine-grain SIMT engine. The benefit of this technique was less apparent
for applications with irregular data-parallelism.

In my paper published at MICRO 2014, I attempted to address the
difficulty of efficiently mapping irregular data-parallel applications to
GPGPUs by utilizing \emph{hardware worklists} to mitigate the classic
challenges involved with double-buffered software worklists: high memory
contention, suboptimal load balancing, and software management
overheads~\cite{kim-hwwl-micro2014}. The hardware worklist is implemented
as per-lane 1rw SRAM \emph{worklist banks} in order to avoid memory
accesses when interacting with the worklist as well as reduce the
software management overhead. The worklist can be accessed with special
instructions for enqueuing/dequeuing tasks. A \emph{worklist
  redistribution} mechanism enables dynamic load balancing of fine-grain
tasks across the worklist banks. At every user-defined sampling interval,
a per-core work redistribution unit moves tasks from banks with more
tasks to banks with less tasks based on the selected policy. Several work
redistribution policies with varying tradeoffs between complexity and
load balancing were explored. A specialized network is used to facilitate
task redistribution between cores.  A \emph{worklist virtualization}
mechanism allowing for seamless spilling/refilling of tasks to/from
memory is used to accommodate tasks that do not fit in the worklist
banks. Using these techniques resulted in speedups between
1.2-2.4$\times$ in irregular data-parallel applications on a cycle-level
model of a modern GPGPU.
